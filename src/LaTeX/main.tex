\documentclass[10pt]{article}
\usepackage[utf8]{inputenc}
\usepackage[spanish]{babel}
\usepackage[a4paper, margin=1in]{geometry}
\usepackage{array}
\usepackage{tikz}
\usepackage{listings}
\usepackage{xcolor}
\usepackage{subfiles}

\definecolor{codegreen}{rgb}{0,0.6,0}
\definecolor{codegray}{rgb}{0.5,0.5,0.5}
\definecolor{codepurple}{rgb}{0.58,0,0.82}
\definecolor{backcolour}{rgb}{0.95,0.95,0.92}

\lstdefinestyle{mystyle}{
    backgroundcolor=\color{backcolour},
    commentstyle=\color{codegreen},
    keywordstyle=\color{magenta},
    numberstyle=\tiny\color{codegray},
    stringstyle=\color{codepurple},
    basicstyle=\ttfamily\footnotesize,
    breakatwhitespace=false,
    breaklines=true,
    captionpos=b,
    keepspaces=true,
    numbers=left,
    numbersep=5pt,
    showspaces=false,
    showstringspaces=false,
    showtabs=false,
    tabsize=2
}

\lstset{style=mystyle}

\renewcommand{\baselinestretch}{1.5}

\begin{document}

    \begin{titlepage}
        \includegraphics[width=2cm]{img/tecnm.png}
        \hfill
        \includegraphics[width=2cm]{img/itsmante.png}
   \begin{center}
        \vspace*{4cm}
        
        \textbf{Analizador Sintáctico}

       \vspace{1cm}
       \includegraphics[width=5cm]{img/isc.png}
       \vspace{1cm}
       

       %\vfill
            
       %\vspace{0.8cm}
     
       %\includegraphics[width=0.4\textwidth]{university}
       Antonio Reyna Espinoza\\     
       Ingeniería en Sistemas Computacionales, Instituto Tecnológico Superior de El Mante\\
       G2201-0096: Lenguajes y Autómatas\\
       M.A.N.M. Verónica Sobrevilla Pintor\\
       \today
            
   \end{center}
\end{titlepage}
    \newpage

    \tableofcontents
    \newpage

    \section{C\'odigo}
    A continuaci\'on se presenta el codigo utilizado para generar un analizador lexico para java utilizando jflex.
    \lstinputlisting[language=Java,label={lst:lexer-code}]{../analyzers/Lexer.flex}
    
    \section{Explicaci\'on}
    Se define el paquete donde se encontrara la clase generada por jflex y se importan las librerias necesarias para la ejecucion.
    \begin{lstlisting}[language=Java,label={lst:lexer-package}]
package model;
import java_cup.runtime.*;
    \end{lstlisting}

\end{document}
